%% naoimi.tex
%% Mac Radigan
\documentclass{article}[11pt]
%\documentclass{report}
\setcounter{secnumdepth}{7}
%\usepackage{draftcopy}
\usepackage[framed,numbered,autolinebreaks,useliterate]{mcode}
\usepackage{setspace}
\usepackage[left=1in,top=1in,right=1in,bottom=1in,nohead]{geometry}
\usepackage{graphicx,amssymb,amstext,amsmath,amsthm,caption,mathtools}
\usepackage{algorithm}
\usepackage{algorithmic}
\usepackage{amsfonts}
\usepackage{amssymb}
\usepackage{csvtools}
\usepackage{pdftexcmds}
\usepackage{listing}
%\usepackage{minted}
\usepackage{fancyvrb}
\usepackage{hyperref}
\usepackage{algpseudocode}
%\usepackage[noend]{algpseudocode}
\hypersetup{
 colorlinks,
 citecolor=blue,
 filecolor=blue,
 linkcolor=blue,
 urlcolor=blue
}
\usepackage{titlesec}
%\titleformat{\section}[block]{\Large\bfseries\filcenter}{}{}{}
%\titleformat{\section}[block]{\Large\bfseries\filcenter}{}{1em}{}
\bibliographystyle{IEEEtran}
\newcommand\Quote[1]{\lq\textsl{#1}\rq}
\newcommand\fr[2]{{\textstyle\frac{#1}{#2}}}
\newcommand{\ssection}[1]{\section[#1]{\centering\normalfont\scshape #1}}
\newcommand{\ssubsection}[1]{\subsection[#1]{\raggedright\normalfont\itshape #1}}
%\addto{\normalsize}{\setlength{\abovedisplayskip}{20ex}}

\begin{document}

\title{Workflow Unified as a single Matrix for Processing Unlimited Services (WUMPUS)}
\author{Mac Radigan}
\date{} % comment this out if you would like to include the date
\doublespacing

\maketitle

\begin{abstract}\centering
...
\end{abstract}

\tableofcontents
\listofalgorithms
%\listoffigures
\newpage

\section{Introduction}

\section{Workflow Unified as a single Matrix for Processing Unlimited Services (WUMPUS)}
%
% \begin{minted}[mathescape,
%                linenos,
%                numbersep=5pt,
%                gobble=2,
%                frame=lines,
%                framesep=2mm]{java}
% // # workflow productions
% // # raw data services
% // $\{d_{y_11}, d_{y_21}, \cdots, d_{y_Q1}\} = \rho_{1}\left(\right)$ # inputs to raw services are implicit (source)
% // $\{d_{y_12}, d_{y_22}, \cdots, d_{y_Q2}\} = \rho_{2}\left(\right)$ # inputs to raw services are implicit (source)
% // $\vdots$
% // # intermediate services
% // $\{d_{y_1k}, d_{y_2k}, \cdots, d_{y_Qk}\} = \rho_{k}\left(\{d_{x_1k}, d_{x_2k}, \cdots, d_{x_Pk}\}\right)$ 
% // $\{d_{y_1k+1}, d_{y_2k+1}, \cdots, d_{y_Qk+1}\} = \rho_{k+1}\left(\{d_{x_1k+1}, d_{x_2k+1}, \cdots, d_{x_Pk+1}\}\right)$ 
% // $\vdots$
% // # final service
% // $\{d_{f_1M}, d_{f_2M}, \cdots, d_{f_QM}\} = \rho_{M}\left(\{d_{x_1M}, d_{x_2M}, \cdots, d_{x_PM}\}\right)$ # this is the prototype service (sink)
% \end{minted}
%
\begin{verbatim}
# workflow productions
# raw data services
\end{verbatim}
$\{d_{y_11}, d_{y_21}, \cdots, d_{y_Q1}\} = \rho_{1}\left(\right)$ \# inputs to raw services are implicit (source)
\newline
$\{d_{y_12}, d_{y_22}, \cdots, d_{y_Q2}\} = \rho_{2}\left(\right)$ \# inputs to raw services are implicit (source)
\newline
$\vdots$
\begin{verbatim}
# intermediate services
\end{verbatim}
$\{d_{y_1k}, d_{y_2k}, \cdots, d_{y_Qk}\} = \rho_{k}\left(\{d_{x_1k}, d_{x_2k}, \cdots, d_{x_Pk}\}\right)$ 
\newline
$\{d_{y_1k+1}, d_{y_2k+1}, \cdots, d_{y_Qk+1}\} = \rho_{k+1}\left(\{d_{x_1k+1}, d_{x_2k+1}, \cdots, d_{x_Pk+1}\}\right)$ 
\newline
$\vdots$
\begin{verbatim}
# final service
\end{verbatim}
$\{d_{f_1M}, d_{f_2M}, \cdots, d_{f_QM}\} = \rho_{M}\left(\{d_{x_1M}, d_{x_2M}, \cdots, d_{x_PM}\}\right)$ \# this is the prototype service (sink)
\newline
%
Thus, each production is a map $f \colon \mathbf{P} \times \mathbf{D}^P \mapsto \mathbf{D}^Q$.  We may re-write the productions as follows:
\newline
\begin{equation}
\begin{split}
\left( \{d_{y_11}, d_{y_21}, \cdots, d_{y_Q1}\} \right) & \leftarrow \left( \rho_{1}, \{\epsilon\} \right) \\
\left( \{d_{y_12}, d_{y_22}, \cdots, d_{y_Q2}\} \right) & \leftarrow \left( \rho_{2}, \{\epsilon\} \right) \\
& \vdots \\
\left( \{d_{y_1k}, d_{y_2k}, \cdots, d_{y_Qk}\} \right) & \leftarrow \left( \rho_{k}, \{d_{x_1k}, d_{x_2k}, \cdots, d_{x_Pk}\} \right) \\
\left( \{d_{y_1k+1}, d_{y_2k+1}, \cdots, d_{y_Qk+1}\} \right) & \leftarrow \left( \rho_{k+1}, \{d_{x_1k+1}, d_{x_2k+1}, \cdots, d_{x_Pk+1}\} \right) \\
& \vdots \\
\left( \{d_{f_1M}, d_{f_2M}, \cdots, d_{f_QM}\} \right) & \leftarrow \left( \rho_{M}, \{d_{x_1M}, d_{x_2M}, \cdots, d_{x_PM}\} \right) 
\end{split}
\label{eq:productionList}
\end{equation}
\newline
%
From the workflow productions we can populate an oriented degree incident matrix, $\mathbb{W}$, having a row for each service, and a column for each Datatype connection between services.  Entries at row i and column j are positive if the $i^{th}$ service takes the $j^{th}$ Datatype as input, and negative if it is an output.  The magnitude of the entry is 1 times the degree of input/output files of the given type.
\newpage
\begin{table}
\centering
\begin{tabular}{1|1 1 1 1}
\toprule
$\mathbb{W}$ & d_{0} & d_{1} & \cdots & d_{N} \\
\hline
\rho{}_{0} & \omega_{1,1}    & \cdots                 & \cdots                 & \omega_{1,N} \\
\rho{}_{1} & \vdots          & \omega_{m,n}           & \reflectbox{$\ddots$}  & \vdots \\
\vdots     & \vdots          & \reflectbox{$\ddots$}  & \ddots                 & \vdots \\
\rho{}_{M} & \omega_{M,1}    & \cdots                 & \cdots                 & \omega_{M,N} \\
\bottomrule
\end{tabular}
\caption{Service Production Oriented Incidence Matrix}
\label{tbl:productionMatrix}
\end{table}
%
\begin{equation}
\omega_{m,n} =
\begin{cases}
-$\alpha$, & \text{if }x = \rho_{m}\left(f_{n}, \cdots\right) \exists x \in \mathbb{D} \mbox{ $\{$ i.e. the file $d_n$ is an input to $\rho{}_m$ with multiplicity $\alpha$ $\}$ } \\
+$\alpha$, & \text{if }d_{n} = \rho_{m}\left(x, \cdots\right) \exists x \in \mathbb{D} \mbox{ $\{$ i.e. the file $d_n$ is an output of $\rho{}_m$ with multiplicity $\alpha$ $\}$ } \\
 0, & \text{otherwise}
\end{cases}
\label{eq:productionCode}
\end{equation}
%
Taking the $sign$ of the matrix we have an ordinary oriented incident matrix, $\mathbb{W}_{u}$.
\begin{equation}
\centering
\mathbb{W}_{u} = signum\left(\mathbb{W}\right)
\label{eq:incidentMatrix}
\end{equation}
\newline
%
We will define the notation $\mathbb{M}\left(i|j\right)}$ to indicate a submatrix of $\mathbb{M}$ having the $i^{th}$ row and $j^{th}$ column removed.  We can then identify the potential starting services matrix, $\mathbb{S}$, as the rows of $\mathbb{W}_{u}$ those which do have any inputs, as shown below.
%
\begin{equation}
\begin{array}{ll} 
\mathbb{S} = \mathbb{W}_{u}\left(|j\right) & \mbox{where } j = \{ x | x \in \mathbb{W}_{u} \land \min(\mathbb{W}_{u})=1 \} \\
\end{array}
\label{eq:intermediateMatrix}
\end{equation}
%
We will designate the final (prototype) service $\mathbb{F}$, and specify that it is known at the time of static analysis.
\newline
Now identify the intermediate services $\mathbb{P}$ knowing that $Intermediate = All \setminus \left( Starting \cup Final \right) $.
%
\begin{equation}
\begin{array}{ll} 
\mathbb{P} = \mathbb{W}_{u}\left(|j\right) & \mbox{where } j = \{ x | x \in \mathbb{W}_{u} \land \min(\mathbb{W}_{u})=1 \lor \max(\mathbb{W}_{u})=-1 \} \\
\end{array}
\label{eq:startingMatrix}
\end{equation}
%
We have thus partitioned $\mathbb{W}_{u}$ into a starting submatrix, $\mathbb{S}$ intermediate submatrix, $\mathbb{P}$ final submatrix, $\mathbb{F}$.
%
\begin{equation}
\mathbb{W}_{u} = \left[
\begin{array}{c} 
\mathbb{S} \\ \hline
\mathbb{P} \\ \hline
\mathbb{F}
\end{array}
\right]
\label{eq:intermediateMatrix}
\end{equation}
%
The Laplacian matrix is the product of an oriented incident matrix, say $\mathbb{M}$,  and its transpose.  Note that this matrix is Hermitian, which is a property we will exploit in the numerical computation of eigenvalues.
\begin{equation}
\mathbb{L} = \mathbb{M}\mathbb{M}^{\intercal}
\label{eq:laplacianMatrix}
\end{equation}
We can determine if all nodes within a submatrix of an oriented incident matrix is connected from its algebraic connectivity.  The second eigenvalue of the Laplacian matrix of a subgraph is the algebraic connectivity, and thus the subgraph is connected if the second eigenvalue is greater than zero.  
\begin{equation} 
\Delta_{\mathbb{L}}\left(s\right) = | s\mathbb{I} - \mathbb{L} |
\label{eq:characteristicPolynomial}
\end{equation}
\begin{equation} 
\underline{\lambda} = \Delta_{\mathbb{L}}\left(0\right)
\label{eq:eigenvalues}
\end{equation}
\begin{equation} 
\lambda_{2} > 0 \Leftrightarrow \mbox{connected}
\label{eq:eigenvalues}
\end{equation}
%
\begin{algorithm}
\caption{Connected}
\label{alg:connected}
\Function{connected}{$\mathbb{M}$}
  \State \Comment{Create the Laplacian matrix}
  \State $\mathbb{L} = \mathbb{M} \times \mathbb{M}^{\intercal}$
  \State \Comment{Solve for the eigenvalues}
  \State $\underline{\lambda} = $\Call{eigW}{$\mathbb{L}$}
  \If{$\lambda_{2}>0$}
    \State \Comment{The second eigenvalue is the algebraic connectivity}
    \State \Return{True}
  \Else
    \State \Return{False}
  \EndIf
\EndFunction
\end{algorithm}
\newline
We may exploit the fact that our Laplacian matrix is Hermitian, and use Kung's Algorithm for \mathbb{QR} factorization (qrDecomp).  We may then use the \mathbb{QR} Algorithm (eigW) to compute the eigenvalues of the Laplacian matrix.
\newline
To support Kung's Algorithm, we introduce a notation for transvections, $T_{i,j}^{N}\left(x\right)$.  The transvection is an $N \times N$ matrix with ones along the diaginal, the value $x$ at row $i$ and column $j$, and zeros elsewhere.
%
\begin{equation}
\centering
\mathbb{T}_{i,j}^{N}\left(x\right) = \mathbb{I}_{N} + x\mathbb{E}_{i,j}
\label{eq:transvection}
\end{equation}
%
Kung's \mathbb{QR} Algorithm then computes a strictly positive diagonalization matrix, $\mathbb{E}$, and resultant diagonal matrix $\mathbb{D}$.  We define $\mathbb{C} = \sqrt{\mathbb{D}}$, and can write the factorization of a matrix $\mathbb{M}_{N \times N}$ as follows.
\begin{equation}
\centering
\mathbb{E} = \displaystyle{\prod_{i=1}^{N}} \displaystyle{\prod_{j \not= i}T_{i,j}^{N}\left(\mathbb{M}_{i,j}\right)^{*}}
\label{eq:qrDecompE}
\end{equation}
\begin{equation}
\centering
\mathbb{D} = \left(\mathbb{EM}\right)^{*}\left(\mathbb{ME}\right) = \mathbb{E}^{*}\mathbb{M}^{*}\mathbb{M}\mathbb{E}
\label{eq:qrDecompD}
\end{equation}
\begin{equation}
\centering
\mathbb{C} = \sqrt{\mathbb{D}}
\label{eq:qrDecompC}
\end{equation}
\begin{equation}
\centering
\mathbb{Q} = \mathbb{A}\mathbb{E}\mathbb{C}^{-1}
\label{eq:qrDecompQ}
\end{equation}
\begin{equation}
\centering
\mathbb{R} = \mathbb{C}\mathbb{E}^{-1}
\label{eq:qrDecompR}
\end{equation}
\newline

\begin{algorithm}
\caption{Transvection}
\label{alg:transvection}
\Function{transvection}{$\mathbb{M}_{N \times N},i,j$}
  \State \Comment{Initialize an identity matrix}
  \State allocate $\mathbb{T}_{N \times N} \gets 0$
  \For{$k \gets 1 \ldots N$}
    \State $\mathbb{T}_{k,k} \gets 1$
  \EndFor
  \State \Comment{Conjugate and copy at i,j}
  \State $\mathbb{T}_{i,j} \gets \mathbb{M}_{i,j}^{*}$
  \State \Comment{Return the transvection}
  \State \Return $\mathbb{T}$
\EndFunction
\end{algorithm}

\begin{algorithm}
\caption{Kung's Algorithm}
\label{alg:qrDecomp}
\Function{qrDecomp}{$\mathbb{M}_{N \ times N}$}
  \State \Comment{Initialize an identity matrix}
  \State allocate $\mathbb{E}_{N \times N} \gets 0$
  \For{$k \gets 1 \ldots N$}
    \State $\mathbb{E}_{k,k} \gets 1$
  \EndFor
  \State \Comment{Create the diagonalizer by applying conjugate pairs of transvections to remove off-diagonal elements}
  \For{$i \gets 1 \ldots N$}
    \For{$j \gets 1 \ldots N$}
      \If{$i \not= j$}
        \State $\mathbb{E} = \mathbb{E} \cdot \Call{Transvection}{\mathbb{M},i,j}$
      \EndIf
    \EndFor
  \EndFor
  \State \Comment{Apply the diagonalizer to create the diagonalization}
  \State $\mathbb{D} \gets \mathbb{E}^{*}\mathbb{M}^{*}\mathbb{M}\mathbb{E}$
  \State $\mathbb{C} \gets \sqrt{\mathbb{D}}$
  \State \Comment{Compute the $\mathbb{Q}$ and $\mathbb{R}$ matrices}
  \State $\mathbb{Q} \gets \mathbb{A}\mathbb{E}\mathbb{C}^{-1}$
  \State $\mathbb{R} \gets \mathbb{C}\mathbb{E}^{-1}$
  \State \Comment{Return the orthogonal and upper triangular decomposition}
  \State \Return $\mathbb{Q}, \mathbb{R}$
\EndFunction
\end{algorithm}

The \mathbb{QR} Algorithm iteratively computes the eigenvalues of a matrix using a \mathbb{QR} decomposition.  The algorithm converges when the elements in the subdiagonal approach zero.

\begin{algorithm}
\caption{QR Algorithm}
\label{alg:eigW}
\Function{eigW}{$\mathbb{A}^{N \times N}_0$}
  \State \Comment{Set the initial conditions}
  \State $k \gets 0$
  \State \mathbb{Q}_0, \mathbb{R}_0 \gets \Call{$qrDecomp$}{$\mathbb{A}_0$}
  \State \Comment{Iterate until the subdiagonal converges to zero}
  \While{$ \mathbb{A}_{k_{i,j}} < \epsilon, \forall j=i-1 \forall i \in 1 \ldots N$ }
    \State $k \gets k+1$
    \State $\mathbb{Q}_{k-1}, \mathbb{R}_{k-1} \gets \Call{qrDecomp}{\mathbb{A}_{k-1}}$
    \State $\mathbb{A}_{k} \gets \mathbb{R}_{k-1}\mathbb{Q}_{k-1}$
  \EndWhile
  \State \Comment{Create a vector from the diagonal elements of $\mathbb{A}_k$}
  \State $\underbar{\lambda} \gets \{ a_{i,j} | \forall a_{i,j} \in \mathbb{A}_k \land i=j \} }$
  \State \Comment{Return eigenvalues}
  \State \Return $\underbar{\lambda}$
\EndFunction
\end{algorithm}

\newpage
\paragraph{Workflow Execution}
%\par
\begin{algorithm}
\caption{On File Processed $\mathbf{CONCURRENT}$}
\label{alg:executeDependency}
\Procedure{onFileProcessed}{fileset, parameters}
  \State \Comment{Compute the state vector}
  \State $\underbar{\mathbb{Z}}_k \gets \Call{stateVector}{fileset}$
  \State \Comment{Compute the state change}
  \State $\Delta\underbar{\mathbb{Z}}_k \gets \underbar{\mathbb{Z}}_k - \underbar{\mathbb{Z}}_{k-1}$
  \State \Comment{Identify the production candidates}
  \State $\underbar{\mathbb{\alpha}}_k \gets \mathbb{W}^{+}\underbar{\mathbb{Z}}_k \oplus \Sigma\mathbb{W}^{+}$
  \State \Comment{Filter the production candidates to yeild the production set}
  \State $\underbar{\mathbb{\rho}} \gets \underbar{\mathbb{\alpha}}_{k} \oplus \Call{sign}{\mathbb{W}^{+}\underbar{\mathbb{Z}}_k}$
  \ForAll{$\rho \in \underbar{\rho}$}
    \State \Comment{Create the input file set}
    \State $\underbar{\mathbb{X}} \gets \mbox{ fileset}\left(\mathbb{W}^{+}_{\underbar{\rho},*}\right)$
    \State \Comment{Call each service in the production set}
    \State fileset.push \Call{$\rho$}{$fileset, parameters$} 
  \EndFor
\EndProcedure
\end{algorithm}
%
The workflow is executed using the $executeWorkflow\left(\right)$ algorithm.
%
\begin{algorithm}
\caption{Execute Workflow}
\label{alg:executeWorkflow}
\Procedure{executeWorkflow}{fileList, parameters}
  \ForAll{$file \in fileList$ \mathbf{ IN PARALLEL}}
    \ForAll{$\rho \in rows(\mathbb{S})$}
      \State \Comment{Construct the augmented graph with starting service (source) $\rho$}
      \State $\mathbb{M} = \left[ \begin{array}{c} \rho \\ \hline \mathbb{P} \\ \hline \mathbb{F} \end{array} \right]$
      \If{\Call{connected}{$\mathbb{M}$} and \Call{$\rho.canCreate$}{x} }
        \State fileset.push \Call{$\rho$}{$file, parameters$} 
      \EndIf
    \EndFor
  \EndFor
\EndProcedure
\end{algorithm}

\section{Summary}

%\section{Appendix A: Algorithms}
%\section{Appendix B: Worked Exercises with Solutions}

\newpage
\bibliography{IEEEabrv,bibliography}

\end{document}
%% *EOF*
